\documentclass[11pt]{article}

\usepackage{fullpage}

\begin{document}

\title{ARM Checkpoint}
\author{Group 19: Harry Uglow, Shayan Khaksar-Ghalati, Maurice Yap and Joshua Cooper}

\maketitle

\section{Group Organisation}

It was decided that Harry would be the leader of this group.\\

We first discussed the design and construction of the main \textit{emulate} function in a meeting on Friday afternoon. Together as a group, we coded the main structure of the emulator, which involved creating structs (in \texttt{structs.h}) to represent our registers and memory, as well as instructions. Instructions would be handled by functions in the \texttt{executeInstructions.c} file.\\

We decided to split up the implementations of these functions by assigning each person an instruction type to concentrate on over the weekend. Because there was an imbalance in the amount of work each person needed to do, those with less work helped those with more work to do to implement their functions. The instruction types were distributed as follows:\\
\begin{itemize}
\item Harry - Single Data Transfer
\item Shayan - Branch
\item Maurice - Multiply
\item Joshua - Data Processing
\end{itemize}
After the weekend, we worked to implement the rest of the main functions of the emulator, including file input, decoding instructions and the pipeline.  Harry focused on the pipeline and handling instructions, while Shayan and Maurice worked on the file input.

\subsection{Communication}
It was agreed that we would all try hard to make all git commit messages as detailed and clear as possible. When not physically working together, Facebook Messenger is our main channel of communication. We use it to discuss what we each need to do and any bugs and faults which need fixing. We also schedule when we come in to labs to work together.

\subsection{Teamwork}
One possible area which we could improve on is better understanding of code written by other members. We could achieve this through individually setting time aside to read through other people's code and ensuring that each of our separate tasks are well commented for others to read and make sense of. This would enable us to work more efficiently in the debugging stage. Fortunately, we have worked very well as a team to debug code quickly.\\

We are also very lucky to have an excellent group leader who has made sure that work is distributed evenly and that everyone keeps on track and understands what they need to do.

\section{Implementation Strategies}
We created the core of the emulator first in the file \texttt{defs.h}. This contained well-defined structures to represent the state of our processor and its inner workings. It was important to do this as it allowed us to complete the implementation of the instruction execution based on this common framework which we all understood.\\

To ensure the readability of our code, we split up each section of the emulator (instruction execution, pipeline, utilities etc.) into separate files which would later be linked to the main \texttt{emulator.c} file through a well organised hierarchy of \texttt{{\#}include} directives.

\subsection{Reusable Code}

We think that we could re-use the \texttt{Instruction} struct, as well as the \texttt{Type} enumeration from the \texttt{defs.h} file used for the emulator. Similar princles will also be used for reading the input file.

\subsection{Foreseeable Challenges}
Macro-processing elements, such as string processing, may pose a challenge, but we shall overcome. We will try to mitigate this problem but carefully designing algorithms for each processing task.





\end{document}
