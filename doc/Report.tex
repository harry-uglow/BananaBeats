\documentclass[11pt]{article}

\usepackage{fullpage}

\begin{document}

\title{Our Extension...}
\author{Group 19: Harry Uglow, Shayan Khaksar-Ghalati, Maurice Yap and Joshua Cooper}

\maketitle

\section{Introduction}

\subsection{Assembler}
\subsubsection{File Structure}
The main function from which the assembler program is run is the sole function in the file, \texttt{assemble.c} and resides in the main \texttt{src/} directory. Its source files are located in the \texttt{assembler/} directory within \texttt{src/}. We have also written a \texttt{Makefile} in the \texttt{src/} directory which compiles both our assembler and emulator with the correct dependencies so that the minimal amount of work required is done in the compilation process.
\subsubsection{Implementation}
We elected to use the two-pass method to implement the assembler. This meant that we needed to create a Symbol Table data type. This simple and minimal data type uses a linked list of label-pointer pairs and includes the \textit{put} and \textit{get} functions which add a label-pointer pair entry to the table and retrieve a pointer given a label argument respectively. There are also functions included to create a new Symbol Table (\textit{new}) and remove a Symbol Table (\textit{delete}) and all its associated nodes from memory. We decided that the second part of the pair would be a generic pointer (\texttt{void *}) as opposed to an integer pointer as this makes the data structure reuseable for purposes other than mapping labels to memory addresses.\\

On the first pass (\texttt{int firstPass(char **argv)} in \texttt{utils.c}), the input file is read and all labels are entered into the Symbol Table, mapped to the address in memory to which each refers. The instructions (i.e. everthing which is not a label) are split up and each part is placed into an array of assembly instruction structs (\texttt{struct assInstr}).\\

The second pass (\texttt{void secondPass()} in \texttt{utils.c}) iterates through the array of assembly instructions. For each one, 


\subsection{Extension plans}

\section{Raspberry Pi Drum Pads???}

\section{Testing}

\section{Reflection}
Group reflection

\subsection{Harry}
Personal reflection

\subsection{Shayan}
Personal reflection

\subsection{Josh}
Personal reflection

\subsection{Maurice}
Personal reflection

\end{document}
